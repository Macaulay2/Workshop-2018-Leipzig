\documentclass[11pt]{extarticle}

% packages

\usepackage{amsfonts,amsthm,latexsym,amsmath,amssymb,amscd,amsmath, mathrsfs, epsf, xypic, geometry}
\usepackage{color}
\usepackage{picture}
\usepackage{tikz}
\usepackage{graphicx}
\usepackage{verbatim}
\usepackage[colorlinks = true]{hyperref}
\geometry{verbose,tmargin=2.5cm,lmargin=2cm,rmargin=2cm}
\headsep 1cm
\renewcommand{\baselinestretch}{1.15}

% THEOREM Environments ---------------------------------------------------
 \newtheorem{theorem}{Theorem}[section]
 \newtheorem{corollary}[theorem]{Corollary}
 \newtheorem{lemma}[theorem]{Lemma}
 \newtheorem{proposition}[theorem]{Proposition}
 \newtheorem{conj}{Conjecture}
 \theoremstyle{definition}
 \newtheorem{definition}[theorem]{Definition}
 \theoremstyle{remark}
 \newtheorem{remark}[theorem]{Remark}
 \theoremstyle{definition}
 \newtheorem{example}[theorem]{Example}
 \newtheorem{question}[theorem]{Question}
  \newtheorem{questions}[theorem]{Questions}

%Math commands

\newcommand{\rk}{\mathrm{rk}}
\newcommand{\elim}{\mathrm{elim}}
\newcommand{\RR}{\mathbb{R}}
\newcommand{\CC}{\mathbb{C}}
\newcommand{\XX}{\mathbb{X}{\bf R}(T)}
\newcommand{\PP}{\mathbb{P}}
\newcommand{\caF}{\mathcal{F}}
\newcommand{\caM}{\mathcal{M}}
\newcommand{\KK}{\mathbb{K}}


\title{{\bf \texttt{Macaulay2} workshop Leipzig 2018 -- project on Tensors}}

%\subjclass[2000]{14P99, 12D05, 13P10, 14A25}

\begin{document}

\maketitle

The main goal of this project is to create a package helpful for researchers interested in {\it tensors} and their {\it decompositions}. In the literature, there are several algorithms (some of them already written in \texttt{Macaulay2}) and new lines of research that should be addressed with tools collected in a well-structured package to be efficiently used. See \cite{Lan12} for an extensive survey on tensors and their uses in algebraic geometry and applications. Here is a list of projects we would like to suggest as points of departure. 



%\bigskip
%We distinguish the difficulty of the exercises/questions by denoting them with 
%
%\centerline{
%* (minimum), ** or *** (maximum). 
%}
%
%\smallskip
%\noindent We suggest the participant not familiar with {\it Waring decompositions of polynomials} and/or {\it Macaulay2} to start with the easier exercises.

\subsection*{Basic constructions of tensors}
{\it Constructions}
\begin{itemize}
\item[(i)] Give an efficient way to handle tensors in {\it Macaulay2}, e.g., how to input a tensor, how to go from a (multi-)homogeneous polynomial and a (partially) symmetric tensor
\end{itemize}

\noindent {\it Manipulations}
\begin{itemize}
\item[(i)] Give functions to construct new tensors from old ones, e.g., Kronecker products, Hadamard products, etc...
\end{itemize}

\subsection*{Apolarity Theory}
\noindent {\it Apolar ideal}
\begin{itemize}
	\item[(i)]  Define a function that, given a homogeneous polynomial $f$, returns the apolar ideal of $f$.
\end{itemize}
{\it Sylvester's Algorithm}
\begin{itemize}
	\item[(i)] Define a function that, given a binary form $f$ returns the ideal of a minimal reduced set of points apolar to $f$.
	\item[(ii)] Define a function that, given a form $f$ in two essential variables returns the ideal of a minimal reduced set of points apolar to $f$.
\end{itemize}

\noindent {\it Apolarity Lemma}
\begin{itemize}
	\item[(i)]  Define a function that, given a homogeneous polynomial $f$ and an ideal $I$, checks if $I$ defines a $0$-dim (reduced) scheme apolar to $f$.
	
	\item[(ii)] Can we define a function that, given a homogeneous polynomial $f$ and an ideal defining a minimal set of reduced points apolar to $f$, returns a decomposition of $f$ using the corresponding linear forms?
	
	OBS.: in general, this might require to work over fields which are not precise... 
\end{itemize}
{\it Generalizations of Apolarity Lemma}
\begin{itemize}
\item[(i)] Apolarity Lemma has been generalized to any {\it toric variety}; see \cite{Gal16, Tei14}. Implement a procedure that, given a toric variety $X$, a closed subscheme $R$ and a non-zero global section $F$, checks if $R$ is apolar to $F$. 

OBS.: as testing cases, we should try to use partially symmetric and non-symmetric tensors.
\end{itemize}

\subsection*{Flattenings: equations of secant varieties and decomposition algorithms}
{\it Flattenings}
\begin{itemize}
	\item[(i)] Define a procedure to construct (symmetric) flattenings of a given (symmetric) tensor.
	OBS.: as testing cases, we should get the equations defining some secant varieties to some Veronese varieties; see the table at page $2$ of \cite{LO13}.
\end{itemize}
{\it Exact decomposition algorithms}
\begin{itemize}
	\item[(i)] Implement the {\it Catalecticant algorithm}; see \cite[Section 12.4.1]{Lan12};
	\item[(ii)] Implement the {\it Koszul flattenings algorithm}; see \cite[Section 12.4.2]{Lan12} and \cite{OO13}. 
The aim of Koszul flattenings is to go beyond lower bounds to border ranks for tensors $T\in A\otimes B\otimes C$ given by rank conditions on flattenings; namely, beyond the equations $\wedge^{r+1} A^{*}\otimes \wedge^{r+1}(B\otimes C)^{*}$. The idea is to use an augmented version of our tensor $T$. Consider $\textnormal{Id}_A\otimes T: A\otimes B^{*}\rightarrow A\otimes A\otimes C$. We show that the flattenings of $\textnormal{Id}_A\otimes T$ have relations with the border rank of $T$. We have the two canonical projections $T^{\wedge}_A: A\otimes B^{*}\rightarrow \wedge^2 A\otimes C$ and $T^s_A: A\otimes B^{*}\rightarrow S^2 A\otimes C$. \\
\indent Suppose $\dim A = {\bf a} = 3$. Choosing bases of $A,B,C$, we may write $T$ as 
$$
T = a_1\otimes X_1+a_2\otimes X_2+a_3\otimes X_3,
$$
\noindent where $X_i : B^{*}\rightarrow C$. In terms of matrices, we have 
$$
T^{\wedge}_A=
\begin{pmatrix}
0 & X_2 & X_3 \\
-X_2 & 0 & -X_1 \\
-X_3 & X_1 & 0\\
\end{pmatrix}.
$$

If the border rank of $T$ satisfies $\underline{{\bf R}}(T)\leq r$, then $\textnormal{rank}(T^{\wedge}_A)\leq r({\bf a}-1)$; hence $r\geq \frac{\textnormal{rank}(T^{\wedge}_A)}{({\bf a}-1)}$. More generally, let $A,B,C$ be vector spaces with ${\bf a}=2p+1\leq {\bf b}\leq {\bf c}$. Consider the map
$$
T^{\wedge p}_A: \wedge^p A\otimes B^{*}\rightarrow \wedge^{p+1} A\otimes C.
$$
\noindent If $T$ has rank one, then $\textnormal{rank}(T^{\wedge p}_A) = \binom{2p}{p}$. If $T$ is generic, then $\textnormal{rank}(T^{\wedge p}_A) = \binom{2p+1}{p}\cdot {\bf b}$. Thus the size $(r+1)\binom{2p}{p}$ minors of $T^{\wedge p}_A$ furnish equations for the $r$th secant variety $\hat{\sigma}_r$ up to $r = \frac{2p+1}{p+1}\cdot{\bf b}-1$. 

OBS.: other procedures to decompose low rank symmetric tensors have been given in \cite{MoOn18}. A package "ApolarLowRank.m2" is also given.

\item[(iii)] Implement the algorithm to find the unique tensor decomposition of a {\it general} element in $\CC^2 \otimes \CC^a \otimes \CC^b$; see \cite[Section 12.4.3]{Lan12}.
	
	OBS.: when these exact algorithms fail to give an exact decomposition, bounds on the ranks of tensors should be obtained. In this direction, e.g., see also \cite[Theorem 3.8.2.4]{Lan12}.
\end{itemize}

\subsection*{Useful constructions to study tensors}

\noindent {\it Symmetry groups of tensors}

\begin{itemize}
\item[]Let $T\in V_1\otimes \cdots \otimes V_d$ be a tensor. Its  {\it symmetry} or {\it isotropy} group is defined to be 
$$
G_T = \lbrace g\in \textnormal{GL}(V_1)\times \cdots \times \textnormal{GL}(V_d)/(\mathbb C^{*})^{d-1} \ | \ gT = T \rbrace.  
$$
\noindent Here the $(d-1)$-dimensional torus comes from the trivial action on every tensor given by $\prod_{i=1}^d \lambda_i = 1$, for $\lambda_i\in \mathbb C^{*}$. Since the action is rational, the symmetry group is an algebraic group and thus an algebraic variety. Given a tensor $T$, we determine the dimension of its symmetry group $G_T$. Moreover, we use \texttt{GAP} to analyze further properties of this group related to its Lie algebra $\mathfrak g_T$. For instance, the Levi-Malcev decomposition of $\mathfrak g_T$ and the structure of its semisimple part. 
\end{itemize}

\bigskip
\noindent {\it Eigenvectors of tensors}

\begin{itemize}
\item[] The \emph{E-eigenvalues} and \emph{E-eigenvectors} of tensors (the ``E'' stands for ``Euclidean'') were proposed independently by Lek-Heng Lim and Liqun Qi in \cite{lim2005singular,qi2005eigenvalues}. There are different types of eigenvectors and eigenvalues in the literature, see \cite{cartwright2013number,hu2013determinants,ni2007degree,qi2007eigenvalues,qi2017tensor}. The notions of E-eigenvalues and E-eigenvectors of tensors arise mainly in the context of approximation of tensors, which deals usually with real tensors. Some excellent references about the \emph{best rank $k$ approximation problem} for tensors are \cite{draisma2017best,friedland2014number}. Possibility to experiment the reality of eigenvectors. Some reality issues of eigenvalues and eigenvectors are studied in \cite{maccioni1972number,kozhasov2017fully}.
\end{itemize}

\noindent  {\it Hyperderminants}

\begin{itemize}
\item[] For definitions and constructions, we refer to \cite[Chapter 14]{GKZ08}.
\end{itemize}

%-----
%
%\begin{itemize}
%
%\item Applications to tensor network states? Interesting algorithms to implement? Probably one interesting thing
%is write down explicitly the tensor and compute ranks of its flattenings. {\color{red}{no idea here, any hints from Anna S.?}}
%\end{itemize}


\bibliographystyle{alpha}
\bibliography{references.bib}

\end{document}

